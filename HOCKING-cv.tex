\documentclass[margin,line]{res}
\usepackage{graphicx}
\usepackage[cm]{fullpage}
\usepackage{hyperref}
\usepackage{url}
\urlstyle{same} 

\usepackage[utf8]{inputenc}
\usepackage[T1]{fontenc}

\oddsidemargin -.5in
\evensidemargin -.5in
\textwidth=6.0in
\itemsep=0in
\parsep=0in

\newenvironment{list1}{
  \begin{list}{\ding{113}}{%
      \setlength{\itemsep}{0in}
      \setlength{\parsep}{0in} \setlength{\parskip}{0in}
      \setlength{\topsep}{0in} \setlength{\partopsep}{0in} 
      \setlength{\leftmargin}{0.17in}}}{\end{list}}
\newenvironment{list2}{
  \begin{list}{$\bullet$}{%
      \setlength{\itemsep}{0in}
      \setlength{\parsep}{0in} \setlength{\parskip}{0in}
      \setlength{\topsep}{0in} \setlength{\partopsep}{0in} 
      \setlength{\leftmargin}{0.2in}}}{\end{list}}

\begin{document}

\name{
\begin{tabular*}{7.4in} {@{\extracolsep{\fill}}lr}
Toby Dylan Hocking & Curriculum Vitae 
\end{tabular*}
}

\begin{resume}
\section{\sc Contact and General Info}
\vspace{.05in}
\begin{tabular*}{6.1in} {@{\extracolsep{\fill}}ll}
 Université de Sherbrooke & Birth: 17 March 1984 in Newport Beach, California\\
  2500 boulevard de l'Université  & Citizenship: USA, Permanent residence: Canada. \\
  Sherbrooke, QC, J1K-2R1 & Languages: English (native), French
                        (fluent since 2009). \\
  E-mail:  toby.hocking@nau.edu & Web: \url{http://tdhock.github.io}, \href{https://tdhock.github.io/blog/2022/erdos-number/}{Erd\H{o}s number = 3}. \\
\end{tabular*}

\section{\sc Research Interests}

Fast, accurate, and interpretable algorithms for learning from large
data, using continuous optimization (clustering, regression, ranking,
classification) and discrete optimization (changepoint detection,
dynamic programming). The main applications for these
algorithms are in genomics, neuroscience, medicine, microbiome,
cybersecurity, robotics, satellite/sonar imagery, climate/carbon
modeling.

\section{\sc Professional Experience \\ \hspace{0.1cm} \\ \includegraphics[width=3cm]{HOCKING-rectangle-lores.jpg}}

{\bf Université de Sherbrooke}, Québec, Canada (2024--present).\\
\vspace*{-.1in}
\begin{list2}
\item[] Tenured Associate Professor, Département d'informatique.
\item[] ``Learning Algorithms, Statistical Software, Optimization (LASSO).''
\end{list2}

{\bf Northern Arizona University}, Flagstaff, Arizona, USA (2018--2024).\\
\vspace*{-.1in}
\begin{list2}
\item[] Tenure-Track Assistant Professor, School of Informatics, Computing, and Cyber Systems.
\item[] ``Optimization algorithms for machine learning and interactive data analysis.''
\end{list2}

{\bf McGill University}, Montreal, Canada (2014--2018).\\
\vspace*{-.1in}
\begin{list2}
\item[] Postdoc with Guillaume Bourque, Department of Human Genetics.
\item[]``Changepoint detection and regression models for peak detection in genomic data.''
\end{list2}

{\bf Tokyo Institute of Technology}, Tokyo, Japan (2013).\\
\vspace*{-.1in}
\begin{list2}
\item[] Postdoc with Masashi Sugiyama, Department of Computer Science.
\item[] ``Support vector machines for ranking and comparing.''
\end{list2}

{\bf Sangamo BioSciences}, Richmond, CA, USA (2006--2008).\\
\vspace*{-.1in}
\begin{list2}
\item[] Research Assistant with Jeff Miller in the Zinc Finger Technology group.
\item[] ``A web app for visualization and statistical analysis of experimental data.''
\end{list2}

\section{\sc Education}

{\bf \'{E}cole Normale Sup\'{e}rieure}, Cachan, France (2009--2012).\\
\vspace*{-.1in}
\begin{list2}
\item[] Math Ph.D. with Francis Bach, D\'{e}partement d'Informatique; Jean-Philippe Vert, Institut Curie.
\item[] ``Learning algorithms and statistical software, with applications to bioinformatics."
\end{list2}

{\bf Universit\'e Paris 6}, Paris, France (2008--2009).\\
\vspace*{-.1in}
\begin{list2}
\item[] Master of Statistics, internship at INRA with Mathieu Gautier and Jean-Louis Foulley.
\item[] ``A Bayesian Outlier Criterion to Detect SNPs under Selection in Large Data Sets."
\end{list2}

{\bf University of California, Berkeley}, CA, USA (2002--2006).\\
\vspace*{-.1in}
\begin{list2}
\item[] Double B.A. in Statistics, Molecular and Cell Biology; thesis in Statistics with Terry Speed.
\item[] ``Chromosomal copy number analysis using SNP microarrays and a binomial test statistic.'' 
\end{list2}

\section{\sc Honors and Awards (Selected)}

Consultant on NIH grant (Autism Data Science Initiative) Sep 2025--Aug 2028, USD\$4,249,998 total, USD\$22,896 my share. ``Advancing Success \& Developmental Outcomes in Autism Spectrum Disorder through Analysis of Secondary Data (ASD3 Outcomes Project)."

PI on grant from Quebec Education Minister (FabriqueREL---free online
educational resources in French), 15,000\$CAD, 2025--2026. ``French
transltion for user manual of animint2 interactive data visualization
R package.''

Project co-lead on grant from UK Medical Research Council (Better
Methods, Better Research program), £50,000, 2025--2028 (my share:
£3,694). ``UKRI1573: Maintaining and expanding the data.table R
package for fast and efficient data manipulation.'' Lead Heather
Turner.

PI on travel grant from DATAIA (Artificial Intelligence Institute of
Université Paris-Saclay), €24,000, Academic year
2024--2025. ``Efficient algorithms and software for change-point
detection.''

PI on National Science Foundation grant 2303612 from program Pathways to Enable Open-Source Ecosystems (POSE), US\$731,881, Sep
2023--Aug 2025. ``POSE: Phase II: Expanding the data.table ecosystem
for efficient big data manipulation in R.'' Co-PIs Igor Steinmacher
and Marco Gerosa.

% https://reporter.nih.gov/project-details/11129998
Co-PI on National Institute of Mental Health grant 3R01MH134177-02S1,
US\$455,660, Aug 2023 to June 2027. ``Addressing Structural Disparities in Autism Spectrum Disorder through Analysis of Secondary Data (ASD3).'' Role: advise PHD student and summer salary for
machine learning analysis of health records related to childhood
autism. US\$40K under my control = US\$10K in each of 4 academic
years.

Senior Personnel on Department of Energy grant, US\$3,600,000, Sept 2022 to
Sept 2025. ``Friends and Foes: microbial interactions and soil
biogeochemistry after 23 years of experimental warming.'' Role: summer
salary for machine learning analysis of microbial interaction data.
US\$30K under my control = US\$10K in each of 3 academic years.

Co-PI on National Science Foundation grant 2125088 from program URoL-Understanding the Rules of Life, US\$3,000,000, Sept 2021
to Aug 2026. ``MIM: Discovering in reverse – using isotopic
translation of omics to reveal ecological interactions in
microbiomes.'' Role: supervise PHD student working on machine learning
analysis of new metabolic flux data, in order to infer new types of
interactions between microbes. US\$200K under my control = US\$40K in
each of 5 academic years.

Senior personnel on NAU US\$819K sub-contract of Department of Energy
grant, US\$9,000,000, 2022-2025, Lawrence Livermore National
Laboratory Science Focus Area program entitled ``Microbes Persist:
Systems Biology of the Soil Microbiome'' led by PI Jennifer
Pett-Ridge. Role: machine learning analysis of microbiome interaction
networks, identifying taxa and traits that are associated with soil
carbon cycling processes. US\$15K under my control = US\$5K in each of
3 academic years.

Senior personnel on Missouri Department of Education grant,
US\$1,509,570, July 2021--June 2023, contract entitled ``MMD-DCI
Research, Development, \& Leadership'' led by PI Ronda Jenson. Role:
summer salary and mentoring graduate students on interpretable machine
learning algorithms for Predictive Modeling Framework for District
Continuous Improvement. US\$50K total under my control = US\$25K in
each of two academic years.

Air Force Research Laboratory, Summer Faculty Fellowship, US\$20,000,
May--July 2021, ``Machine learning algorithms for understanding
physically unclonable functions based on resistive memory devices.''

PI on R Consortium Grant, US\$34,000, Jan--Dec 2020, ``RcppDeepState: an easy
way to fuzz test compiled code in R packages.''

``Mobilit\'e entrant'' travel award, research about dynamic
programming algorithms for constrained change-point detection, with
Guillem Rigaill in Universit\'e Evry, France, 2016.

International useR conference, Best Student Poster Award, ``Adding
direct labels to plots,'' 2011.

INRIA/INRA (French computer science and agricultural research institutes), Ph.D. scholarship, 2009 (declined).

UC Berkeley, Department of Statistics VIGRE research scholarship, 2005.

\section{\sc Papers in progress and under review}

Rust K, {\bf Hocking TD}. A Log-linear Gradient Descent Algorithm for
Unbalanced Binary Classification using the All Pairs Squared Hinge
Loss. Under review in {\it Journal of Machine Learning Research},
arXiv:2302.11062.

\section{\sc Peer-reviewed journal papers}

Agyapong D, Propster JR, Marks J, and {\bf Hocking TD}. Cross-validation for training and testing co-occurrence network inference algorithms. {\it BMC Bioinformatics}, 26(74), 2025.

Nguyen TL and {\bf Hocking TD}. Penalty Learning for Optimal Partitioning using Multilayer Perceptron. {\it Statistics and Computing}, 35, 2025.

Gurney KR, Aslam B, Dass P, Gawuc L, {\bf Hocking TD}, Barber JJ, and Kato A. Assessment of the climate trace global powerplant co2 emissions. {\it Environmental Research Letters}, 19(11):114062, oct 2024.

Kaufman J, Stenberg A, {\bf Hocking TD}. Functional Labeled Optimal
Partitioning. {\it Journal of Computational and Graphical Statistics},
DOI:10.1080/10618600.2023.2293216. (2024)

Bodine CS, {\bf Hocking TD}, Buscombe D. Automated river substrate
mapping from sonar imagery with machine learning.  {\it
  Journal of Geophysical Research --- Machine Learning and
  Computation} 1(3). (2024)

Tao F, Houlton BZ, Frey SD, Lehmann J, Manzoni S, Huang Y, Jiang L,
Mishra U, Hungate BA, Schmidt MWI, Reichstein M, Carvalhais N, Ciais
P, Wang Y-P, Ahrens B, Hugelius G, {\bf Hocking TD}, Lu X, Shi Z, Viatkin K,
Vargas R, Yigini Y, Omuto C, Malik AA, Peralta G, Cuevas-Corona R, Di
Paolo LE, Luotto I, Liao C, Liang Y-S, Saynes VS, Huang X, Luo
Y. Reply to: Model uncertainty obscures major driver of soil
carbon. {\it Nature} (2024), volume 627, pages E4-E6.

Tao F, Huang Y, Hungate BA, Manzoni S, Frey SD, Schmidt MWI,
Reichstein M, Carvalhais N, Ciais P, Jiang L, Lehmann J, Mishra U,
Hugelius G, {\bf Hocking TD}, Lu X, Shi Z, Viatkin K, Vargas R, Yigini
Y, Omuto C, Malik AA, Peralta G, Cuevas-Corona R, Di Paolo LE, Luotto
I, Liao C, Liang YS, Saynes VS, Huang X, Luo Y. Microbial carbon use
efficiency promoting global soil carbon storage. {\it Nature} (2023).
DOI:10.1038/s41586-023-06042-3.

Hillman J, {\bf Hocking TD}. Optimizing ROC Curves with a Sort-Based
Surrogate Loss Function for Binary Classification and Changepoint
Detection. {\it Journal of Machine Learning Research} 24(70):1-24, 2023.

Runge V, {\bf Hocking TD}, Romano G, Afghah F, Fearnhead P, Rigaill
G. gfpop: an R Package for Univariate Graph-Constrained Change-point
Detection. {\it Journal of Statistical Software} (2023), Volume 106,
Issue 6.

Harshe K, Williams J, {\bf Hocking TD}, Lerner Z. Predicting Neuromuscular
Engagement to Improve Gait Training with a Robotic Ankle
Exoskeleton. {\it IEEE Robotics and Automation Letters}, vol. 8,
no. 8, pp. 5055-5060, Aug. 2023, doi: 10.1109/LRA.2023.3291919.

{\bf Hocking TD}, Srivastava A. Labeled Optimal Partitioning. {\it
  Computational Statistics} DOI: 10.1007/ s00180-022-01238-z (2022).

Mihaljevic JR, Borkovec S, Ratnavale S, \textbf{Hocking TD}, Banister
KE, Eppinger JE, Hepp CM, Doerry E. SPARSEMODr: Rapidly simulate
spatially explicit and stochastic models of COVID-19 and other
infectious diseases. {\it Biology Methods \& Protocols}, Volume 7,
Issue 1, 2022.

Barnwal A, Cho H, {\bf Hocking TD}. Survival regression with
accelerated failure time model in XGBoost. {\it Journal of
  Computational and Graphical Statistics} (2022).

{\bf Hocking TD}, Rigaill G, Fearnhead P, Bourque G. Generalized
Functional Pruning Optimal Partitioning (GFPOP) for Constrained
Changepoint Detection in Genomic Data. {\it Journal of Statistical
  Software}, Vol. 101, Issue 10 (2022).

Chaves AP, Egbert J, {\bf Hocking TD}, Doerry E, Gerosa MA. Chatbots
language design: the influence of language use on user
experience. {\it ACM Transactions on Computer-Human Interaction} 29,
2, Article 13 (2022).

{\bf Hocking TD}, Vargovich J. Linear Time Dynamic Programming for
Computing Breakpoints in the Regularization Path of Models Selected
From a Finite Set. {\it Journal of Computational and Graphical
  Statistics} (2021), doi:10.1080/10618600.2021.2000422.

{\bf Hocking TD}. Wide-to-tall data reshaping using regular
expressions and the nc package. {\it R Journal} (2021),
doi:10.32614/RJ-2021-029.

Liehrmann A, Rigaill G, {\bf Hocking TD}. Increased peak detection
accuracy in over-dispersed ChIP-seq data with supervised segmentation
models. {\it BMC Bioinformatics} 22, Article number: 323 (2021).

Fotoohinasab A, {\bf Hocking TD}, Afghah F. A Greedy Graph Search
Algorithm Based on Changepoint Analysis for Automatic QRS-Complex
Detection. {\it Computers in Biology and Medicine} 130 (2021).

Abraham A, Prys-Jones T, De Cuyper A, Ridenour C, Hempson G, {\bf
  Hocking TD}, Clauss M, Doughty C. Improved estimation of gut passage
time considerably affects trait-based dispersal models. {\it
  Functional Ecology} (2021); 35: 860-869.

{\bf Hocking TD}, Rigaill G, Fearnhead P, Bourque G. Constrained
dynamic programming and supervised penalty learning algorithms for
peak detection in genomic data. {\it Journal of Machine Learning
  Research} 21(87):1--40, (2020).

{\bf Hocking TD}. Comparing namedCapture with other R packages for
regular expressions. {\it R Journal} (2019). doi:10.32614/RJ-2019-050

Jewell S, {\bf Hocking TD}, Fearnhead P, Witten D. Fast Nonconvex
Deconvolution of Calcium Imaging Data. {\it Biostatistics} (2019), doi:
10.1093/biostatistics/kxy083.

Depuydt P, Koster J, Boeva V, {\bf Hocking TD}, Speleman F,
Schleiermacher G, De Preter K. Meta-mining of copy number profiles of
high-risk neuroblastoma tumors. {\it Scientific Data} (2018).

Alirezaie N, Kernohan KD, Hartley T, Majewski J, {\bf Hocking
  TD}. ClinPred: Prediction Tool to Identify Disease-Relevant
Nonsynonymous Single-Nucleotide Variants. American Journal of Human
Genetics (2018). doi:10.1016/j.ajhg.2018.08.005

Sievert C, Cai J, VanderPlas S, Khan F, Ferris K, {\bf Hocking
  TD}. Extending ggplot2 for linked and dynamic web graphics. {\it
  Journal of Computational and Graphical Statistics} (2018).

Depuydt P, Boeva V, {\bf Hocking TD}, {\it et al}. Genomic
Amplifications and Distal 6q Loss: Novel Markers for Poor Survival in
High-risk Neuroblastoma Patients. {\it Journal of the National Cancer
  Institute} (2018). DOI:10.1093/jnci/djy022.

{\bf Hocking TD}, Goerner-Potvin P, Morin A, Shao X, Pastinen T,
Bourque G. Optimizing ChIP-seq peak detectors using visual labels and
supervised machine learning. {\it Bioinformatics} (2017) 33 (4): 491-499.

Shimada K, Shimada S, Sugimoto K, Nakatochi M, Suguro M, Hirakawa A,
{\bf Hocking TD}, Takeuchi I, Tokunaga T, Takagi Y, Sakamoto A, Aoki T, Naoe
T, Nakamura S, Hayakawa F, Seto M, Tomita A, Kiyoi H. Development and
analysis of patient-derived xenograft mouse models in intravascular
large B-cell lymphoma. {\it Leukemia} (2016).

Chicard M, Boyault S, Colmet-Daage L, Richer W, Gentien D, Pierron G,
Lapouble E, Bellini A, Clement N, Iacono I, Bréjon S, Carrere M, Reyes
C, {\bf Hocking TD}, Bernard V, Peuchmaur M, Corradini N, Faure-Conter
C, Coze C, Plantaz D, Defachelles A-S, Thebaud E, Gambart M, Millot F,
Valteau-Couanet D, Michon J, Puisieux A, Delattre O, Combaret V,
Schleiermacher G. Genomic copy number profiling using circulating free
tumor DNA highlights heterogeneity in neuroblastoma. {\it Clinical Cancer
Research} (2016).

Maidstone R, {\bf Hocking TD}, Rigaill G, Fearnhead P. On optimal
multiple changepoint algorithms for large data. {\it Statistics and
Computing} (2016). doi:10.1007/s11222-016-9636-3 

Suguro M, Yoshida N, Umino A, Kato H, Tagawa H, Nakagawa M, Fukuhara
N, Karnan S, Takeuchi I, {\bf Hocking TD}, Arita K, Karube K, Tsuzuki
S, Nakamura S, Kinoshita T, Seto M. Clonal heterogeneity of lymphoid
malignancies correlates with poor prognosis. {\it Cancer Sci.} (2014)
Jul;105(7):897-904.

{\bf Hocking TD}, Boeva V, Rigaill G, Schleiermacher G,
Janoueix-Lerosey I, Delattre O, Richer W, Bourdeaut F, Suguro M, Seto
M, Bach F, Vert J-P. SegAnnDB: interactive Web-based genomic
segmentation. {\it Bioinformatics} (2014) 30 (11):
1539-1546. DOI:10.1093/bioinformatics/btu072

{\bf Hocking TD}, Wutzler T, Ponting K and Grosjean P. Sustainable,
extensible documentation generation using inlinedocs. {\it Journal of
Statistical Software} (2013), 54(6), 1-20. DOI:10.18637/jss.v054.i06

{\bf Hocking TD}, Schleiermacher G, Janoueix-Lerosey I, Boeva V, Cappo
J, Delattre O, Bach F, Vert J-P. Learning smoothing models of copy
number profiles using breakpoint annotations. {\it BMC Bioinfo.} (2013),
14:164. DOI:10.1186/1471-2105-14-164

Gautier M, {\bf Hocking TD}, Foulley JL. A Bayesian outlier criterion
to detect SNPs under selection in large data sets. {\it PloS ONE} 5
(8), e11913 (2010).

Doyon Y, McCammon JM, Miller JC, Faraji F, Ngo C, Katibah GE, Amora R,
{\bf Hocking TD}, Zhang L, Rebar EJ, Gregory PD, Urnov FD, Amacher
SL. Heritable targeted gene disruption in zebrafish using designed
zinc-finger nucleases. {\it Nature biotechnology} 26 (6), 702-70
(2008).

\section{\sc Peer-reviewed conference papers}

In addition to peer-reviewed journals, I publish papers at highly
competitive computer science conferences like {\it ICML} and {\it
  NeurIPS}, with double-blind peer reviews, and $\approx$20\%
acceptance rates.

Barr J, {\bf Hocking TD}, Morton G, Thatcher T, Shaw P. Classifying
Imbalanced Data with AUM Loss. 2022 Fourth International Conference on
Transdisciplinary AI (TransAI). 

{\bf Hocking TD}, Barr J, Thatcher T. Interpretable linear models for predicting security vulnerabilities in source code. 2022 Fourth International Conference on Transdisciplinary AI (TransAI). 

Barr J, Shaw P, Abu-Khzam FN, Thatcher T, {\bf Hocking TD}. Graph
Embedding: A Methodological Survey. 2022 Fourth International
Conference on Transdisciplinary AI (TransAI). 

Kolla AC, Groce A, {\bf Hocking TD}. Fuzz Testing the Compiled Code in
R Packages. IEEE 32nd International Symposium on Software
Reliability Engineering (ISSRE 2021), pp. 300-308, doi:
10.1109/ISSRE52982.2021.00040.

Fotoohinasab A, {\bf Hocking TD}, Afghah F. A Graph-Constrained
Changepoint Learning Approach for Automatic QRS-Complex
Detection. {\it Asilomar Conference on Signals, Systems, and
  Computers} (2020).

Fotoohinasab A, {\bf Hocking TD}, Afghah F. A Graph-constrained
Changepoint Detection Approach for ECG Segmentation. 
{\it 42th Annual International Conference of the IEEE Engineering in
  Medicine and Biology Society (EMBC 2020)}.

{\bf Hocking TD}, Bourque G. Machine Learning Algorithms for
Simultaneous Supervised Detection of Peaks in Multiple Samples and
Cell Types. {\it Pacific Symposium on Biocomputing} 25:367-378 (2020).

Drouin A, {\bf Hocking TD}, Laviolette F. Maximum margin interval
trees. {\it Neural Information Processing Systems (NeurIPS)}, 2017.

{\bf Hocking TD}, Rigaill G, Bourque G. PeakSeg: constrained optimal
segmentation and supervised penalty learning for peak detection in
count data. {\it International Conference on Machine Learning (ICML)},
2015.

{\bf Hocking TD}, Rigaill G, Bach F, Vert J-P. Learning sparse
penalties for change-point detection using max-margin interval
regression. {\it International Conference on Machine Learning (ICML)}, 2013.

{\bf Hocking TD}, Joulin A, Bach F, Vert J-P. Clusterpath: an
Algorithm for Clustering using Convex Fusion Penalties. {\it International Conference on Machine Learning (ICML)}, 2011.

\section{\sc Books, Chapters, Manuals}

{\bf Hocking TD} and Killick R. {\it Changepoint detection algorithms
  and applications in R}. Textbook in preparation.

{\bf Hocking TD}. Introduction to Machine Learning and Neural
Networks. Chapter in textbook {\it Land Carbon Cycle Modeling: Matrix
  Approach, Data Assimilation, and Ecological Forecasting}, edited by
Yiqi Luo, published in 2022 by Taylor and Francis.

{\bf Hocking TD}. Animated interactive data visualization using the
grammar of graphics (The animint2 Manual), 17 web pages/chapters with
interactive graphics and exercises. (2018)

\section{\sc Conference Tutorials}

{\bf Hocking TD}, Killick R. Introduction to optimal changepoint
detection algorithms, {\it useR} 2017.

{\bf Hocking TD}, Ekstr\o m CT. Understanding and creating interactive
graphics, {\it useR} 2016.

\section{\sc Invited talks (selected)}

ASU Data-oriented Mathematical and Statistical Sciences seminar, Joint
Statistical Meetings Toronto, Universit\'e Laval, Universit\'e
Sherbrooke (2023); Institute of Mathematical Statistics London, IEEE
conference in Prescott Arizona, University of Arizona Math/Stats
seminar (2022); ASU West ML Day,
\href{https://arizona.hosted.panopto.com/Panopto/Pages/Viewer.aspx?id=4e87c8d0-96d2-40d1-808c-ad16014c6962}{TRIPODS
  University of Arizona}, IEEE NJACS (2021); NAU Math Department
Colloquium (2018); University of Waterloo, Université de Montréal,
Sainte-Justine Children's Hospital, University of Québec à Montréal,
Polytechnique Montréal (2017); Universit\'e Laval (2016); McGill
Barbados epigenomics workshop (2015); Sapporo Japan Workshop on
Machine Learning and Applications to Biology (2013); Google Research
New York, Universit\'e Rennes, Universit\'e Angers, INRIA Lille
(2012); Institut de Biologie de Lille (2011).

\section{\sc Consulting (selected)}

Acronis SCS, cybersecurity company in Phoenix (2022). Interpretable
and non-linear machine learning algorithms which use source code
analysis to predict software vulnerabilities.

Plotly, data visualization startup in Montreal (2015). Original author
of ggplot functionality in plotly R package.

\section{\sc Teaching}

All of my course materials are freely available online,
\url{https://tdhock.github.io/teaching/}

Fall 2023, Northern Arizona University, CS 470, Artificial Intelligence.

Fall 2023, Northern Arizona University, CS 599, Deep Learning.

Fall 2023, Northern Arizona University, CS 599, Unsupervised Learning.

Spring 2023, 2 hour lecture ``Introduction to Deep Learning in R'' for
graduate student training ``Research Bazaar Arizona.''

Spring 2023, Northern Arizona University, CS470, Artificial Intelligence.

Spring 2023, Northern Arizona University, CS105/205/305, Computing Tools.

Fall 2022, Northern Arizona University, CS499, Deep Learning.

Fall 2022, Northern Arizona University, CS105/205/305, Computing Tools.

Spring 2022, Northern Arizona University, CS570, Deep Learning.

Fall 2021, Northern Arizona University, CS499/599, Unsupervised Learning.

Summer 2021, 1 hour lecture ``Introduction to Machine Learning and
Neural Networks'' for summer school on ``New Advances in Land Carbon
Cycle Modeling.''

Spring 2021, Northern Arizona University, CS470, Artificial Intelligence.

Fall 2020, Northern Arizona University, CS499/599, Unsupervised
Learning.

Summer 2020, 90 minute lecture ``Introduction to Machine Learning and
Neural Networks'' for summer school on ``New Advances in Land Carbon
Cycle Modeling.''

Spring 2020, Northern Arizona University, CS499, Deep Learning.

Fall 2019, Northern Arizona University, CS/EE599, Reproducible Machine
Learning Research.

Spring 2019, Northern Arizona University, CS499, Optimization
algorithms for machine learning.

\section{\sc Professional Service}

2023--present, Associate Editor for the journal Stat.

2023--present, co-author of Omics CRAN Task View.

2023, co-organized WNAR session on changepoint detection, with Ning
Hao and Selena Niu.

2022, organized two topic contributed sessions about changepoint
detection for Institute of Mathematical Statistics meeting in London.

2022, Member of steering committee for R project in Google Season of Docs.

2021--present, machine learning editor for rOpenSci Statistical Software.

2021--present, co-author of
\href{https://contributor.r-project.org/rdevguide/}{R Development
  Guide} and member of R Contribution Working Group (resources for
making it easy/accessible to contribute improvements to base R).

2018--present, editor for Journal of Statistical Software.

2017--2018, president of organizing committee for ``R in Montreal
2018'' conference.

2012--present, co-administrator and mentor for R project in Google
Summer of Code (teaching free/open-source software development, how to
create and improve R packages).

2010--present: peer reviewer for Technometrics, International
Conference on Machine Learning (ICML), Advances in Neural Information
Processing Systems (NeurIPS), Journal of Machine Learning Research
(JMLR), Artificial Intelligence Review, Journal of Computational and
Graphical Statistics (JCGS), R Journal, Bioinformatics, PLOS
Computational Biology, BMC Bioinformatics, IEEE Transactions on
Pattern Analysis and Machine Intelligence, Information and Inference,
Journal of Statistical Computation and Simulation, Computo, Genome
Biology.

\section{\sc Software Online (Selected)} 

Numerous original/novel free/open-source software contributions using R, C, C++,
Python, and JavaScript. Available from standard repositories (CRAN for
R, PyPI for Python) as well as GitHub:
\url{https://github.com/tdhock}, links to my software contributions
\url{https://tdhock.github.io/software/}

{\bf R}: contributions to base R regex functionalty and data reshaping
in data.table package. Maintainer of numerous R packages (17 on CRAN
as of Feb 2022) for machine learning (changepoint detection,
classification, regression, ranking, etc), directlabels for labeled
figures, animint2 for animated interactive figures, inlinedocs for
documentation generation.

{\bf Python}: contributions to pandas module for data manipulation
(str.extractall regex functionality), maintainer of GUI/web server
software for labeling and changepoint detection in genomic data
(annotate\_regions, SegAnnDB, PeakLearner).

% Minimum qualifications:
% An earned Ph.D. in Computer Science, Data Science, or a closely-related discipline, by the time of appointment; an outstanding research record; an ability to lead a research program involving students and postdocs; an ability to teach effectively at the undergraduate and graduate levels; and a strong potential for securing external funding.

% Desirable Qualifications: Experience with data-science issues and applications in areas related to climate science, digitized/regenerative economies, smart cities.

\section{\sc References}
 
\noindent {\bf Bruce Hungate}, collaborator on research papers and grants related to machine learning for ecological data analysis. \\
Regents' Professor, Biological Sciences, Northern Arizona University\\
E-mail: Bruce.Hungate@nau.edu. Telephone: 928-523-0925

\noindent {\bf Yiqi Luo}, collaborator on research and teaching related to machine learning for climate science (summer school, textbook).\\
Professor, School of Integrative Plant Science, Cornell University\\
E-mail: yl2735@cornell.edu

\noindent {\bf Olivia Lindly}, collaborator on research papers and grants related to interpretable machine learning algorithms for understanding autism.\\
Associate Professor, Department of Health Sciences, Northern Arizona University\\
Phone: +1 (928) 523-5155, E-mail: Olivia.Lindly@nau.edu}

\noindent {\bf Jarrett ``Jay'' Barber}, colleague familiar with my teaching related to statistical machine learning.\\
Associate Professor, School of Informatics, Computing and Cyber Systems,\\
Northern Arizona University, 
Phone: +1 (928) 523-6869, E-mail: Jarrett.Barber@nau.edu

\noindent {\bf Alex Groce}, collaborator on research papers and grants.\\
Associate Professor, School of Informatics, Computing and Cyber Systems,\\
Northern Arizona University, 
Phone: +1 (928) 523-8263, E-mail: Alex.Groce@nau.edu

\noindent {\bf Fatemeh Afghah}, collaborator on research papers and grants.\\
Associate Professor, Clemson University, Phone: +1 (864) 656-0100, E-mail: fafghah@clemson.edu, Web: \url{https://fafghah.people.clemson.edu/}

\noindent {\bf Guillem Rigaill}, collaborator on research papers and grants.\\
Researcher at INRAE (French Agronomy Research Institute)\\
Phone: +33 (0) 1 64 85 35 44, E-mail: guillem.rigaill@inrae.fr\\
Web: \url{http://www.math-evry.cnrs.fr/members/Grigaill/welcome}

\noindent {\bf Rebecca Killick}, collaborator on a conference tutorial.\\
Professor of Statistics at Lancaster University\\
Phone: +44 (0)1524 593780, E-mail: r.killick@lancaster.ac.uk\\
Web: \url{http://www.lancs.ac.uk/~killick/}

\end{resume}
\end{document}




